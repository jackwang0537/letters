\documentclass[11pt, oneside]{letter}   	% use "amsart" instead of "article" for AMSLaTeX format
\usepackage{geometry}                		% See geometry.pdf to learn the layout options. There are lots.
\geometry{letterpaper}                   		% ... or a4paper or a5paper or ... 
%\geometry{landscape}                		% Activate for rotated page geometry
%\usepackage[parfill]{parskip}    		% Activate to begin paragraphs with an empty line rather than an indent
\usepackage{graphicx}				% Use pdf, png, jpg, or eps§ with pdflatex; use eps in DVI mode
								% TeX will automatically convert eps --> pdf in pdflatex		
\usepackage{amssymb}
\usepackage{indentfirst}

%SetFonts

%SetFonts

\begin{document}

\signature{Father}

\address{26 Broadway \\New York}
\date{November 19, 1887}

\begin{letter}{}%{Company name \\ Street\\ City\\ Country}

\opening{Dear Johnny:}
\setlength{\parindent}{0.5cm}

Yours, 17th, just at hand, and so happy to hear from you and that you and your little Mother are doing so nicely in the quiet of the woods. Am in the midst of hard battles today, but getting on nicely, and looking forward to dropping in upon you sometime before long. It is not likely I can do so, however, for Thanksgiving, which I regr et. I have been called to Washington, but did not heed the summons, but hold myself in readiness to go at a moment's notice. Mr. Brewster and several others go over to be there Monday morning. Think your decision right in reference to the shed. Everything going on nicely at home. Old Mr. Hubbell, now 77 years of age, came down this morning. He is the same dear good Methodist man he used to be. Stayed with us at the table and drank coffee and ate cakes and syrup. Went over the house and enjoyed it much. Gave him an order for a bedstead for the spare room. His price will be several hundred dollars less than Pottier and I presume the bedstead will be as good or better than Pottier's, and we know the beds are unsurpassed. Should this not be satisfactory to Mamma, we can have the bedstead sent to Forest Hill for next year, as we have talked of having new ones there. We like the appearance of the new man and think he will do well. Everything goes on smoothly in the house. You say you are not lonely. That is not the case with me. I am however doing the best I can, and the rest are all doing remarkably well.

I have not written to Mama but telegraphed every day. Been so busy and knew numbers of letters were written from the house. She will get all the news. Business affairs going nicely. With much love to you and Mama and great appreciating of the beautiful letter you wrote me.

\closing{Your loving}

\end{letter}

%\ps{P.S. Here goes your ps.}
%\encl{Enclosures.}

\begin{letter}{}%{Company name \\ Street\\ City\\ Country}

\setlength{\parindent}{0cm}

\opening{MR.John D Rockefeller,jr.\\Forest Hill,\\East Cleveland,O.}

\setlength{\parindent}{0.5cm}
BY 1887 the John D.Rocefell family had two homes in Cleveland and one in New York City.Rokefeller purchased 997 Euclid Avenue,Cleveland,in1868.It was a two-story Victorian home.Johnn D.Rockefeller,Jr.was bron in this house on January 29,1874.The second Cleveland home,Forest Hill,eventually a700-acre estate,was acquired in 1878.The house,a three-story building,was originally built as a health resort.Over the years Rockefeller developed the property with a farm,two lakes for skating and swimming,a racetrck for his horses,bicycle trails,and a nine-hole golf course.The house burned down in December 1917.In the 1930s part of the estate was developded into single-family homes,an apartment complex,and a shopping area.The rest was given to the public as park.

Between 1877 and 1884,when in New York,Rokefeller and his family had rooms in Buckingham Hotel,between 49th and 50th Streets on Fifth Avenue.In October 1884 he purchased 4 West 54th Streets a furnished four-story brownstone house built in 1865-66.He made few changes in the housed during the four decades he lived there.

After 1877 the family normally spent the months from May to October in Cleveland and October to May in New York.However,Junior and his mother, Laura Spelman Rockefeller,spent the winter of 1887-88 at Forest Hill,as Junior was not in the best of health.This separation led to the earliest surviving correapondence between Rockefeller and his son.

In the fall of 1887 and the spring of 1888 Rockefeller and other trustees of the Standard Oil Trust tesrifide before the U.S.House Committee investigating trusts.Benjamin Brewster was one of the original trustees.

Pottier was a highly regarded New York manufactuer and retailer of furniture.

\end{letter}

\begin{letter}{}%{Company name \\ Street\\ City\\ Country}

\address{26 Broadway \\New York}
\date{November 28th 1887}

\opening{Dear Johnny:}
\setlength{\parindent}{0.5cm}
Yous,of the 22nd,duly receivd.Excuse delay in answering.Have also your telegram of today for the cutter,and will attend to it tomorrow morning.I assume you want the one to carry two persons.I had a pleasant time in Washington.It is a beautiful city.The weather was mild and lovely.After receiving my testimony they did not wish any other although they had subpoenaed eight of us.We feel very well about the experience over there The Now York World hasn't any further amunition in this direction,is now going back to its first love,the Buffalo suit,trying to rake up something against us.Had a delightful Sunday at home yesterday.Feeling well and ready for business.Looking forward with pleasure to seeing you the last of this week.
Concur in your decision about painting the storm doors.You and Mother will surely have your own way in all these affairs,what's the use of my saying a word.You are monarch of all you survey.


\end{letter}



\end{document}


























